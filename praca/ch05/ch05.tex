Powyższa praca miała na celu zbudowanie oraz wdrożenie reaktywnego intranetu dla firmy z sektora IT. Aplikacja powstała z wykorzystaniem najnowszych trendów oraz technologi w dziedzinie wytwarzana aplikacji sieciowych. Nowoczesne aplikacji sieciowe coraz bardziej upodobniają się do ich biurkowych odpowiedników. Głównie dotycz to interfejsu użytkownika oraz sposobu jego działania, który to na przestrzeni kilku ostatnich lata podlegał dużym zmianom. 

Dziś, nowoczesna aplikacja sieciowa to pojedyncza strona WWW, której to elementy podlegają zmianą w odpowiedzi na działanie użytkownika. Komunikacja z serwerem odbywa się z wykorzystaniem API, w większości typu REST. Po stronie klienta odbywa się buforowanie danych tak aby interakcja pomiędzy aplikacją i użytkownikiem odbywała się jak najbardziej płynie. Duża ilość obliczeń, walidacji danych, przetwarzania danych odbywa się po stronie klienta. Aplikacje takie unikają przeładowań całej zawartości strony. 

Opisane w pracy oraz pokazane w postaci aplikacji techniki, technologie oraz rozwiązania spełniają wszystkie wymagania jakie stawia się dziś nowoczesnej aplikacji sieciowej. Jedyny wyjątkiem jest komunikacja z serwerem. Wykorzystując framework Meteor.js w miejsce np REST'owego API dostajemy coś bardziej nowoczesnego oraz o wiele ciekawszego. Opierając się na DDP\footnote{Distributed Data Protocol} zyskujemy mniej praco oraz czasochłonny  sposób na synchronizację danych pomiędzy serwerem a klientem, który w działaniu jest bardzo spektakularny. Nie musimy pisać kodu synchronizującego dane, framework sam to realizuje. Meteor.js wpisuje się w trend używania jednego języka programowania po stronie serwera jaki i klienta.

Wdrożenie aplikacji jest w wstępnej fazie. Aplikacja podlega testom wykonywanym przez pracowników firmy. Po usunięciu błędów, dodaniu brakujących elementów aplikacji oraz po uzupełnieniu dokumentacji aplikacja zostanie wdrożona z wykorzystaniem serwera współdzielonego lub dedykowanego. 