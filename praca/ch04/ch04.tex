Aplikacje napisane z użyciem frameworka Meteor.js mają dość specyficzne wymagania co do serwera na jakim będą uruchamiane. Meteor.js oferuję swoją własną implementacje MongoDB. Jednak do celów produkcyjnych nie jest wskazane wykorzystanie tego rozwiązania. Z tego powodu hosting musi posiadać normalną wersję bazy MongoDB. Jedynym gotowym takim rozwiązaniem jest hosting oferowany przez twórców Meteor.js. Druga opcja to serwer współdzielony, bądź dedykowany, na którym mielibyśmy pełną kontrolę. Jako system operacyjny najlepiej oraz najtaniej jest wykorzystać, którąś z popularnych dystrybucji Linuxa. Najlepsze efekty można uzyskać wykorzystując osobne serwery na aplikację oraz bazę danych. Do tego celu można wykorzystać wirtualizację z wykorzystaniem Dockera\footnote{Otwarte oprogramowanie służące jako „platforma dla programistów i administratorów do tworzenia, wdrażania i uruchamiania aplikacji rozproszonych”. Docker jest określany jako narzędzie, które pozwala umieścić program oraz jego zależności w lekkim, przenośnym, wirtualnym kontenerze, który można uruchomić na prawie każdym serwerze z systemem Linux}, opartego na tak zwanych kontenerach (\textit{linux containers}). 

Pierwszym etapem wdrożenia była instalacja aplikacji z wykorzystaniem rozwiązania oferowanego przez twórców frameworka. Hosting ten nie oferuje zbyt dużej wydajności, aplikacje po określonym czasie są usypiane i potrzebują około 60 sekund na ponowne uruchomienie. Przez małą wydajność doskonale sprawdza się jako platforma testowa co do optymalizacji aplikacji oraz działania w środowisku dużych opóźnień pomiędzy serwerem a klientami. Hosting aplikacji jest darmowy, jednak wymaga rejestracji darmowego konta, a do uruchomienia aplikacji wykorzystuje się gotowe, dostarczane przez framework rozwiązanie. W celu uruchomienia aplikacji na serwerze używamy polecenia \verb|meteor deploy| uruchamianego w głównym katalogu aplikacji. Listing \ref{lst:meteor_deploy} pokazuje użycie tego polecenia. 
\begin{bash}[caption={Uruchomienie aplikacji na serwerze},label={lst:meteor_deploy}]
meteor deploy intranet.yp2.meteor.com --settings server/config/serverConfig.json
\end{bash}
Jako parametr obowiązkowy musimy przekazać pod domenę pod jaką ma działać aplikacja. Dodatkowo możemy przekazać plik z ustawieniami wykorzystując \verb|--settings|.

Aktualnie aplikacja jest w fazie testowania przez pracowników firmy. Naprawiane są zgłaszane błędy. Aplikacji brakuje także kilku kluczowych elementów jak na przykład zarządzanie użytkownikami w organizacji, panel do administrowania całością aplikacji, ustawienia użytkowników. Brakuje także dokumentacji, która jest niezbędna do poprawnego wdrożenia aplikacji.

Następne etapy wdrożenia aplikacji nastąpią po usunięciu zgłoszonych błędów, dodania wyżej wymienionych kluczowych elementów oraz sporządzeniu dokumentacji. Planowane jest wykupienie serwera współdzielonego, który będzie najtańszą opcją. Jako system operacyjny dla serwera zostanie wykorzystana dystrybucja Linuxa - Debian. Uruchomienie aplikacji w tym środowisku planowane jest z wykorzystaniem Dockera. Rozwiązanie to pozwoli w bardzo prosty oraz szybki sposób przenieść aplikacje na serwer o wyższych parametrach. Aplikacja oraz baza danych zostaną podzielone na odrębne kontenery, co dodatkowo podniesie skalowalność aplikacji.   
