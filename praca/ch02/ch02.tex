Rozdział ten przedstawia wykorzystane technologie oraz języki programowania użyte podczas projektowania oraz programowania aplikacji Intranet. Aplikacja powstała z wykorzystaniem \emph{JavaScript}'u, framework'a aplikacji sieciowych \emph{Meteor.js}, nierelacyjnej bazy danych  \emph{MongoDB} oraz framework'a CSS \emph{Bootstrap}, \emph{HTML5}, \emph{CSS3}, \emph{less} --- dynamiczny jeżyk arkuszy stylów oraz gotowy szablon dla panelu administracyjnego \emph{AdminLTE} wykorzystujący Bootstrap.

\section{JavaScript}

Język programowania JavaScript został użyty do zaprogramowania zarówno części serwerowej (\textit{back-end}) jaki i części odpowiedzialnej za interakcje z użytkownikiem (\textit{front-end}) --- interfejs użytkownika. Obecne strony WWW a w szczególności aplikacje dostępne przez przeglądarkę (Gmail, Google Docs, Google Maps, Facebook) szeroko korzystają z JavaScript w celu dostarczenie wielofunkcyjnego oraz interaktywnego interfejsu użytkownika. Jednym z powodów wykorzystanie JavaScript była możliwości wykorzystania go po stornie serwera oraz klienta. Najpopularniejszy obecnie sposób tworzenia stron/aplikacji internetowych wyróżnia trzy warstwy --- warstwę struktury (HTML), warstwę prezentacji (CSS) oraz warstwę zachowania (JavaScript)\cite{stefanov10}.

Internet powstał jako zbiór statycznych dokumentów HTML, które były powiązane hiperłączami. Po wzroście popularności oraz rozmiaru sieci, autorom stron przestały wystarczać dostępne narzędzia. Widoczna stała się potrzeba poprawienia interakcji z użytkownikiem. U jej podstaw leżała chęć zmniejszenie ilości zapytań serwerów w celu realizowania prostych zadań takich jak walidacja formularzy. W tym czasie pojawiły się dwie możliwości - aplety Javy oraz język \emph{LiveScript}, który został zaproponowany przez firmę Netscape w roku 1995. Został on dołączony do przeglądarki Netscape 2.0 pod nazwą JavaScript\cite{stefanov10}.  