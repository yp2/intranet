\documentclass{beamer}

\usepackage[utf8]{inputenc}
\usepackage[MeX]{polski}
\usepackage{graphicx}
\usepackage{mathtools}
\usepackage{tabularx}
\usepackage{multicol}

\usepackage{xcolor}
\usepackage{listings}
\lstset{basicstyle=\ttfamily,
  showstringspaces=false,
  commentstyle=\color{red},
  keywordstyle=\color{blue},
  backgroundcolor=\color{white},
  tabsize=4
}

\usetheme{Warsaw}

\graphicspath{ {gfx/} }

\title[]{Projekt oraz wdrożenie reaktywnego intranetu dla firmy – wykorzystanie technologi Meteor.js, Bootstrap i MongoDB}
\author{Daniel Dereziński}
%\subtitle{Promotor:  prof. dr hab. Grzegorz Wójcik}
\institute[]{Promotor:  prof. dr hab. Grzegorz Wójcik\\[12pt] Wyższa Szkoła Przedsiębiorczości i Innowacji w Lublinie}
\date{\today}


\AtBeginDocument{%
\renewcommand{\raggedright}{\leftskip=0pt \rightskip=0pt}%
}
\newenvironment{justbe}%
{\setlength{\leftmargini}{0pt}\begin{itemize}\item[]}%
{\end{itemize}}

\begin{document}
%----------- slide --------------------------------------------------%
\begin{frame}
\titlepage
\end{frame}

\begin{frame}
\frametitle{}
\footnotesize
\tableofcontents
\end{frame}

\section{Wstęp}
	\subsection{Cel Pracy}

%----------- slide --------------------------------------------------%
\begin{frame}
	\frametitle{Cel pracy}
	\begin{justbe}
 	Celem pracy jest zaprojektowanie, zaprogramowanie oraz wdrożenie reaktywnego intranetu dla firmy z wykorzystaniem technologii Meteor.js, Boostrap oraz nieralacyjnej bazy danych MongoDB.
	\end{justbe}
\end{frame}

	\subsection{Intranet}
%----------- slide --------------------------------------------------%
\begin{frame}
	\frametitle{Intranet}
	\begin{justbe}
		\textbf{Intranet} to sieć komputerowa ograniczająca się do komputerów w np. przedsiębiorstwie lub innej organizacji. Umożliwia korzystanie w obrębie sieci LAN z usług takich, jak strony WWW, poczta elektroniczna itp. czyli usług typowo internetowych. Do intranetu dostęp mają zazwyczaj tylko pracownicy danej organizacji.
	\end{justbe}
\end{frame}


%----------- slide --------------------------------------------------%
\begin{frame}
	\frametitle{}
	\begin{justbe}
		Intranet przypomina internet, z tym jednak zastrzeżeniem, że jest ograniczony do wąskiej grupy osób.\\
		\\
		Kiedyś intranety działały tylko w zamkniętych sieciach wewnętrznych firm. W obecnych czasach intranet wychodzi poza firmę,
	\end{justbe}
\end{frame}

%----------- slide --------------------------------------------------%
\begin{frame}
	\frametitle{}
\end{frame}

%----------- slide --------------------------------------------------%
\begin{frame}
	\frametitle{}
\end{frame}

%----------- slide --------------------------------------------------%
\begin{frame}
	\frametitle{}
\end{frame}

%----------- slide --------------------------------------------------%
\begin{frame}
	\frametitle{}
\end{frame}

%----------- slide --------------------------------------------------%
\begin{frame}
	\frametitle{}
\end{frame}

%----------- slide --------------------------------------------------%
\begin{frame}
	\frametitle{}
\end{frame}

%----------- slide --------------------------------------------------%
\begin{frame}
	\frametitle{}
\end{frame}

%----------- slide --------------------------------------------------%
\begin{frame}
	\frametitle{}
\end{frame}

%----------- slide --------------------------------------------------%
\begin{frame}
	\frametitle{}
\end{frame}



\end{document}