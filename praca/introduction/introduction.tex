\section{O aplikacjach sieciowych}
opisać z \cite{strack15}
\section{Intranet i jego historia}

Intranet jest to sieć komputerowa ograniczająca się do komputerów np.\ w danym przedsiębiorstwie lub innej organizacji, dostępna wyłącznie dla
pracowników danej organizacji. Intranet dostarcza szeroki zakres informacji oraz usług z~wewnętrznych systemów IT organizacji, które nie są
dostępne z~publicznego --- zewnętrznego --- Internetu. Firmowy Intranet dostarcza między innymi centralny punkt wewnętrznej komunikacji, współpracy. 
Intranet stanowi także pojedynczy punkt dostępu do wewnętrznych jakich zewnętrznych zasobów organizacji. W najprostszej formie intranet budowany
jest z~wykorzystaniem sieci typu \emph{LAN} (sieć lokalna) oraz \emph{WAN} (rozległa sieć komputerowa)\cite{intranetWiki}.

Coś o historii intranetu....

\section{Intranet obecnie oraz jego zastosowania}


\section{Cel pracy}

W dzisiejszych czasach wiele organizacji / firm wykorzystuje w swojej działalności z jakieś formy intranetu --- komunikacja, praca zespołowa. 
Są to rozwiązania oparte o darmowe systemy CMS, komunikatory, kalendarze, systemy do zarządzania zadaniami. Celem niniejszej pracy było opracowanie
oraz wdrożenie systemu intranetowego dla firmy zajmującej się produkcją oprogramowania. Firmy z tej branży pracują w oparciu o projekty. Jednym z
podstawowych celów intranetu jest wymiana wiedzy oraz komunikacja. Zaprojektowana oraz zaprogramowana aplikacja umożliwia dodawanie artykułów,
dodawania kategorii, dodawanie projektów, komunikację w obrębie projektów wraz z możliwością dodawania artykułów. Aplikacja pozwala utworzyć profil 
dla organizacji, zaprosić użytkowników do tak utworzonego profilu w celu podjęcia wspólnej pracy. Możliwe jest także tworzenie kont dla użytkowników
nie powiązanych z żadną organizacją.

\section{Struktura pracy}
W rodziałe 2 zostanie przedstawione ....

