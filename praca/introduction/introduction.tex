\section{Intranet}
Intranet jest to sieć komputerowa ograniczająca się do komputerów na przykład w~danym przedsiębiorstwie lub innej organizacji, dostępna dla jej pracowników. Intranet dostarcza szeroki zakres informacji oraz usług z~wewnętrznych systemów IT organizacji,~które nie są dostępne z~publicznego --- zewnętrznego --- Internetu. Firmowy Intranet dostarcza między innymi centralny punkt wewnętrznej komunikacji i współpracy. Intranet stanowi także pojedynczy punkt dostępu do wewnętrznych jak i zewnętrznych zasobów organizacji. W najprostszej formie intranet budowany jest z~wykorzystaniem sieci typu \emph{LAN} (sieć lokalna) oraz \emph{WAN} (rozległa sieć komputerowa)\cite{intranetWiki}.

Pierwsze wdrożenia intranetu miały miejsce już w 1994 roku, które realizowane były przez duże organizacje. Wypuszczenie w 1995 roku przez Microsoft darmowego serwera WWW przyczyniło się do upowszechnienia tej technologi przez szeroki rynek.

\section{Intranet obecnie}
Obecnie systemy intranetowe przekształcają się platformę do dostarczania narzędzi,~są to między innymi narzędzia do współpracy pomiędzy pracownikami, zaawansowane narzędzia wymiany plików, narzędzia do zarządzania relacjami z~klientami, narzędzia do zarządzania projektami czy czasem pracy. Jest to tylko mały wycinek tego co oferują współczesne systemy. Intranet jest także platformą do wymiany wiedzy, pomysłów i idei przez co podnosi produktywność, twórczość pracowników oraz pozwala wypracowywać nowe pomysły oraz wytyczać nowe cele dla organizacji. Obecne systemy wydzielają także część sowich zasobów dla klientów organizacji w celu poprawy jakości oraz szybkości komunikacji pomiędzy pracownikami a klientami. Udostępnianie klientom ważnych informacji o produktach oferowanych przez organizację pozwala na odciążenie pracowników wsparcia klienta. Intranet pozwala także nie tworzenie miejsc pracy w domu poprzez udostępnianie, przez publiczny Internet, zasobów dostępnych w wewnętrznej sieci organizacji. 

\section{Cel pracy}

W dzisiejszych czasach wiele organizacji oraz firm korzysta z jakieś formy intranetu. W~większości są to rozwiązania oparte o darmowe systemy CMS, komunikatory, kalendarze, systemy do zarządzania zadaniami. Celem niniejszej pracy było opracowanie oraz wdrożenie systemu intranetowego dla firmy zajmującej się produkcją oprogramowania. Firmy z tej branży pracują w oparciu o projekty. Jednym z podstawowych celów intranetu jest wymiana wiedzy oraz komunikacja. Zaprojektowana oraz zaprogramowana aplikacja umożliwia dodawanie artykułów, dodawania kategorii, dodawanie projektów, komunikację w obrębie projektów wraz z możliwością dodawania artykułów. Aplikacja pozwala utworzyć profil dla organizacji, zaprosić użytkowników do tak utworzonego profilu w celu podjęcia wspólnej pracy. Możliwe jest także tworzenie kont dla użytkowników nie powiązanych z żadną organizacją.

\section{Struktura pracy}
W rozdziale drugim zostanie przedstawiona część teoretyczna opisująca technologie, rozwiązania oraz techniki wykorzystane podczas prac na projektem aplikacji Intranet. Rozdział trzeci przedstawia założenia jakie miała spełniać aplikacja, opisuje ciekawe oraz interesujące miejsca implementacji. W rozdziale czwarty omawiany jest aktualny etap wdrożenia aplikacji. Rozdział piąty stanowi podsumowania niniejszej pracy.